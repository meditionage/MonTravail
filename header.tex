%Da Latex für englischsprachige Texte ausgerichtet ist,
%wird als Dokumentenklasse das "`scrbook"' von Markus Kohm verwendet.
%Dieses ist für deutschsprachige Texte ausgelegt.
%BCOR12mm: 12mm Bindekorrektur (Verbreiterung des linken Randes)
%DIV11: entspricht in etwas der geforderten Textgrösse und Seitenränder
%titlepage: eine Titelseite wird verwendet
%a4paper: DIN A4
%oneside: für eine spätere einseitige Bedruckung 
\documentclass[BCOR12mm,DIV11,titlepage,a4paper,oneside]{scrbook}

%Paket für deutsche Silbentrennung etc.
%\usepackage{ngerman}
\usepackage[ngerman]{babel}
\usepackage[latin1]{inputenc}
%\usepackage[Encodierer]{inputenc}

%Paket für Zeichenkodierung, entspricht UTF-8
%\usepackage[utf8x]{inputenc}
%(Tabellenfunktionen)
%\usepackage{booktabs}
%(Sonderschriftzeichen).
%\usepackage{pxfonts}

%Paket das die Ausgabefonts definiert
\usepackage[T1]{fontenc}

%Mathepakete
\usepackage{amsfonts}
\usepackage{amsmath}
\usepackage{cancel}
\usepackage{mathcomp}

\usepackage{pdfpages}

%\usepackage[options]{subfigure}

%Paket für das Einbinden von Grafiken über die figure-Umgebung
\usepackage{graphicx}
\usepackage{multirow}
\usepackage{array}


%Paket zum ändern der Kopf- und Fusszeile
\usepackage{fancyhdr}
%Benutzt das Paket für eigenen Seitenstil
\pagestyle{fancy} 
%Erzeugt eine Linie in der Kopfzeile (lässt sich mit 0.0pt ausblenden)
\renewcommand*{\headrulewidth}{0.4pt} 
\lhead{} %Kopfzeile links
\chead{} %Kopfzeile mitte
\rhead{\thepage} %Kopfzeile rechts
\lfoot{} %Fusszeile links
\cfoot{} %Fusszeile mitte
\rfoot{} %Fusszeile rechts
%ändert die Seitennummerierung beim Inhaltsverzeichnis mit eigenem Stil
\renewcommand*{\indexpagestyle}{fancy}
%Verhindert die Seitennummerierung auf den Part-Seiten
\renewcommand*{\partpagestyle}{empty}
%ändert die Seitennummerierung bei Chapter mit eigenem Stil
\renewcommand*{\chapterpagestyle}{fancy}


%Abbildungsnummerierung ändern (abhägig von chapter, z.B. Abbildung 1.1)
\renewcommand*{\thefigure}{\thechapter.\arabic{figure}}
%Tabellennummerierung ändern (abhängig von chapter, z.B. Tabelle 1.1)
\renewcommand*{\thetable}{\thechapter.\arabic{table}}

%Paket zur präfix änderung in Verzeichnissen

\usepackage[titles]{tocloft}

%Formatierung

%Paket, um ein Glossar/Abkürzungsverzeichnis anzulegen
\usepackage{nomencl}
\let\abbrev\nomenclature
%Der Name wird in Glossar geändert
\renewcommand{\nomname}{Abkürzungsverzeichnis}
%Definiert die Aufteilung im Glossar zwischen Begriffen und Erläuterung
\setlength{\nomlabelwidth}{.15\hsize}
%Definiert die Punktelinien im Glossar
%\renewcommand{\nomlabel}[1]{#1 \dotfill}
\setlength{\nomitemsep}{-\parsep}
%Veranlasst die Erstellung des Glossars
\makenomenclature

\setlength{\parindent}{0pt}

\usepackage{makeidx}
\makeindex

%Verhindern, dass eine neue Seite für ein einzelnes Wort/Zeile verwendet wird
\clubpenalty = 10000 % schliesst Schusterjungen aus 
\widowpenalty = 10000 % schliesst Hurenkinder aus (keine Beleidigung, sondern wirklich ein Fachbegriff)

%Paket für ein deutsches Literaturverzeichnis
\usepackage{bibgerm}

%Paket für die Verwendung von URLs durch den Befehl \url{}
\usepackage{url}

%Paket für Zeilenabstand (onehalfspace, singlespace)
\usepackage{setspace}

%Paket zur Erzeugung von Anführungszeichen durch \enquote{Text}
\usepackage[ngerman]{babel}
\usepackage[babel, german=quotes]{csquotes}

%Paket für farbigen Text
%black,white,green,red,blue,yellow,cyan,magenta
\usepackage{color}

\usepackage{colortbl}

%Paket für farbigen Hintergrund für Verbatim-Umgebung (Quelltext-Umgebung)
\usepackage{fancyvrb}
\usepackage{verbatim,moreverb}
%Grauton für Quelltext-Umgebung definieren 80% Grau
\definecolor{sourcegray}{gray}{.80}
%Paket für Quelltext-Umgebung
\usepackage{listings}

%Paket für Positionierung der Objekte ohne Float (Verwendungsbsp.: \begin{figure}[H])
\usepackage{float}

%Paket zur Erzeugung von Hyperrefs und PDF Informationen
\usepackage[pdftex,plainpages=false,pdfpagelabels,
            pdftitle={Diplomarbeit},
            pdfauthor={Thomas Karanatsios}
            ]{hyperref}