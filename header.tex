%Da Latex f�r englischsprachige Texte ausgerichtet ist,
%wird als Dokumentenklasse das "`scrbook"' von Markus Kohm verwendet.
%Dieses ist f�r deutschsprachige Texte ausgelegt.
%BCOR12mm: 12mm Bindekorrektur (Verbreiterung des linken Randes)
%DIV11: entspricht in etwas der geforderten Textgr�sse und Seitenr�nder
%titlepage: eine Titelseite wird verwendet
%a4paper: DIN A4
%oneside: f�r eine sp�tere einseitige Bedruckung 
\documentclass[BCOR12mm,DIV11,titlepage,a4paper,oneside]{scrbook}

%Paket f�r deutsche Silbentrennung etc.
\usepackage{ngerman}
%\usepackage[latin1]{inputenc}

%Paket f�r Zeichenkodierung, entspricht UTF-8
%\usepackage[utf8x]{inputenc}
%(Tabellenfunktionen)
%\usepackage{booktabs}
%(Sonderschriftzeichen).
%\usepackage{pxfonts}

%Paket das die Ausgabefonts definiert
\usepackage[T1]{fontenc}

%Mathepakete
\usepackage{amsfonts}
\usepackage{amsmath}
\usepackage{cancel}
\usepackage{mathcomp}

\usepackage{pdfpages}

%\usepackage[options]{subfigure}

%Paket f�r das Einbinden von Grafiken �ber die figure-Umgebung
\usepackage{graphicx}
\usepackage{multirow}
\usepackage{array}


%Paket zum �ndern der Kopf- und Fusszeile
\usepackage{fancyhdr}
%Benutzt das Paket f�r eigenen Seitenstil
\pagestyle{fancy} 
%Erzeugt eine Linie in der Kopfzeile (l�sst sich mit 0.0pt ausblenden)
\renewcommand*{\headrulewidth}{0.4pt} 
\lhead{} %Kopfzeile links
\chead{} %Kopfzeile mitte
\rhead{\thepage} %Kopfzeile rechts
\lfoot{} %Fusszeile links
\cfoot{} %Fusszeile mitte
\rfoot{} %Fusszeile rechts
%�ndert die Seitennummerierung beim Inhaltsverzeichnis mit eigenem Stil
\renewcommand*{\indexpagestyle}{fancy}
%Verhindert die Seitennummerierung auf den Part-Seiten
\renewcommand*{\partpagestyle}{empty}
%�ndert die Seitennummerierung bei Chapter mit eigenem Stil
\renewcommand*{\chapterpagestyle}{fancy}


%Abbildungsnummerierung �ndern (abh�gig von chapter, z.B. Abbildung 1.1)
\renewcommand*{\thefigure}{\thechapter.\arabic{figure}}
%Tabellennummerierung �ndern (abh�ngig von chapter, z.B. Tabelle 1.1)
\renewcommand*{\thetable}{\thechapter.\arabic{table}}

%Paket zur pr�fix �nderung in Verzeichnissen

\usepackage[titles]{tocloft}

%Formatierung

%Paket, um ein Glossar/Abk�rzungsverzeichnis anzulegen
\usepackage{nomencl}
\let\abbrev\nomenclature
%Der Name wird in Glossar ge�ndert
\renewcommand{\nomname}{Abk�rzungsverzeichnis}
%Definiert die Aufteilung im Glossar zwischen Begriffen und Erl�uterung
\setlength{\nomlabelwidth}{.15\hsize}
%Definiert die Punktelinien im Glossar
%\renewcommand{\nomlabel}[1]{#1 \dotfill}
\setlength{\nomitemsep}{-\parsep}
%Veranlasst die Erstellung des Glossars
\makenomenclature

\setlength{\parindent}{0pt}

\usepackage{makeidx}
\makeindex

%Verhindern, dass eine neue Seite f�r ein einzelnes Wort/Zeile verwendet wird
\clubpenalty = 10000 % schliesst Schusterjungen aus 
\widowpenalty = 10000 % schliesst Hurenkinder aus (keine Beleidigung, sondern wirklich ein Fachbegriff)

%Paket f�r ein deutsches Literaturverzeichnis
%\usepackage{bibgerm}
%Paket f�r ein deutsches Literaturverzeichnis Contents lists
\usepackage[ngerman, english]{babel}

%Paket f�r die Verwendung von URLs durch den Befehl \url{}
\usepackage{url}

%Paket f�r Zeilenabstand (onehalfspace, singlespace)
\usepackage{setspace}

%Paket zur Erzeugung von Anf�hrungszeichen durch \enquote{Text}
\usepackage[ngerman]{babel}
\usepackage[babel, german=quotes]{csquotes}

%Paket f�r farbigen Text
%black,white,green,red,blue,yellow,cyan,magenta
\usepackage{color}

\usepackage{colortbl}

%Paket f�r farbigen Hintergrund f�r Verbatim-Umgebung (Quelltext-Umgebung)
\usepackage{fancyvrb}
\usepackage{verbatim,moreverb}
%Grauton f�r Quelltext-Umgebung definieren 80% Grau
\definecolor{sourcegray}{gray}{.80}
%Paket f�r Quelltext-Umgebung
\usepackage{listings}

%Paket f�r Positionierung der Objekte ohne Float (Verwendungsbsp.: \begin{figure}[H])
\usepackage{float}

%Paket zur Erzeugung von Hyperrefs und PDF Informationen
\usepackage[pdftex,plainpages=false,pdfpagelabels,
            pdftitle={Masterarbeit},
            pdfauthor={El Mehdi Bennani}
            ]{hyperref}