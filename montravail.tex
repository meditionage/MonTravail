%================================================================================
% Erstellt am: 				01.09.2015
% Aufberarbeitet am:	01.09.2015
% Autor:							El Mehdi Bennani
%================================================================================

%Einbinden der Datei header.tex; diese enthält alle verwendeten Pakete,
%sowie Änderungen am Layout
%Da Latex f�r englischsprachige Texte ausgerichtet ist,
%wird als Dokumentenklasse das "`scrbook"' von Markus Kohm verwendet.
%Dieses ist f�r deutschsprachige Texte ausgelegt.
%BCOR12mm: 12mm Bindekorrektur (Verbreiterung des linken Randes)
%DIV11: entspricht in etwas der geforderten Textgr�sse und Seitenr�nder
%titlepage: eine Titelseite wird verwendet
%a4paper: DIN A4
%oneside: f�r eine sp�tere einseitige Bedruckung 
\documentclass[BCOR12mm,DIV11,titlepage,a4paper,oneside]{scrbook}

%Paket f�r deutsche Silbentrennung etc.
\usepackage{ngerman}
%\usepackage[latin1]{inputenc}

%Paket f�r Zeichenkodierung, entspricht UTF-8
%\usepackage[utf8x]{inputenc}
%(Tabellenfunktionen)
%\usepackage{booktabs}
%(Sonderschriftzeichen).
%\usepackage{pxfonts}

%Paket das die Ausgabefonts definiert
\usepackage[T1]{fontenc}

%Mathepakete
\usepackage{amsfonts}
\usepackage{amsmath}
\usepackage{cancel}
\usepackage{mathcomp}

\usepackage{pdfpages}

%\usepackage[options]{subfigure}

%Paket f�r das Einbinden von Grafiken �ber die figure-Umgebung
\usepackage{graphicx}
\usepackage{multirow}
\usepackage{array}


%Paket zum �ndern der Kopf- und Fusszeile
\usepackage{fancyhdr}
%Benutzt das Paket f�r eigenen Seitenstil
\pagestyle{fancy} 
%Erzeugt eine Linie in der Kopfzeile (l�sst sich mit 0.0pt ausblenden)
\renewcommand*{\headrulewidth}{0.4pt} 
\lhead{} %Kopfzeile links
\chead{} %Kopfzeile mitte
\rhead{\thepage} %Kopfzeile rechts
\lfoot{} %Fusszeile links
\cfoot{} %Fusszeile mitte
\rfoot{} %Fusszeile rechts
%�ndert die Seitennummerierung beim Inhaltsverzeichnis mit eigenem Stil
\renewcommand*{\indexpagestyle}{fancy}
%Verhindert die Seitennummerierung auf den Part-Seiten
\renewcommand*{\partpagestyle}{empty}
%�ndert die Seitennummerierung bei Chapter mit eigenem Stil
\renewcommand*{\chapterpagestyle}{fancy}


%Abbildungsnummerierung �ndern (abh�gig von chapter, z.B. Abbildung 1.1)
\renewcommand*{\thefigure}{\thechapter.\arabic{figure}}
%Tabellennummerierung �ndern (abh�ngig von chapter, z.B. Tabelle 1.1)
\renewcommand*{\thetable}{\thechapter.\arabic{table}}

%Paket zur pr�fix �nderung in Verzeichnissen

\usepackage[titles]{tocloft}

%Formatierung

%Paket, um ein Glossar/Abk�rzungsverzeichnis anzulegen
\usepackage{nomencl}
\let\abbrev\nomenclature
%Der Name wird in Glossar ge�ndert
\renewcommand{\nomname}{Abk�rzungsverzeichnis}
%Definiert die Aufteilung im Glossar zwischen Begriffen und Erl�uterung
\setlength{\nomlabelwidth}{.15\hsize}
%Definiert die Punktelinien im Glossar
%\renewcommand{\nomlabel}[1]{#1 \dotfill}
\setlength{\nomitemsep}{-\parsep}
%Veranlasst die Erstellung des Glossars
\makenomenclature

\setlength{\parindent}{0pt}

\usepackage{makeidx}
\makeindex

%Verhindern, dass eine neue Seite f�r ein einzelnes Wort/Zeile verwendet wird
\clubpenalty = 10000 % schliesst Schusterjungen aus 
\widowpenalty = 10000 % schliesst Hurenkinder aus (keine Beleidigung, sondern wirklich ein Fachbegriff)

%Paket f�r ein deutsches Literaturverzeichnis
%\usepackage{bibgerm}
%Paket f�r ein deutsches Literaturverzeichnis Contents lists
\usepackage[ngerman, english]{babel}

%Paket f�r die Verwendung von URLs durch den Befehl \url{}
\usepackage{url}

%Paket f�r Zeilenabstand (onehalfspace, singlespace)
\usepackage{setspace}

%Paket zur Erzeugung von Anf�hrungszeichen durch \enquote{Text}
\usepackage[ngerman]{babel}
\usepackage[babel, german=quotes]{csquotes}

%Paket f�r farbigen Text
%black,white,green,red,blue,yellow,cyan,magenta
\usepackage{color}

\usepackage{colortbl}

%Paket f�r farbigen Hintergrund f�r Verbatim-Umgebung (Quelltext-Umgebung)
\usepackage{fancyvrb}
\usepackage{verbatim,moreverb}
%Grauton f�r Quelltext-Umgebung definieren 80% Grau
\definecolor{sourcegray}{gray}{.80}
%Paket f�r Quelltext-Umgebung
\usepackage{listings}

%Paket f�r Positionierung der Objekte ohne Float (Verwendungsbsp.: \begin{figure}[H])
\usepackage{float}

%Paket zur Erzeugung von Hyperrefs und PDF Informationen
\usepackage[pdftex,plainpages=false,pdfpagelabels,
            pdftitle={Masterarbeit},
            pdfauthor={El Mehdi Bennani}
            ]{hyperref} 

\begin{document}

%=== Einleitung ======================================================
%Seitennummerierung Abstract bis einschliesslich Inhaltsverzeichnis
\frontmatter 
%Seitenzahlzähler wird auf 3 gesetzt, da Titelseiten nicht mitgezählt werden, 0 ist besser
%Seitennummerierung findet in arabischen Zahlen statt 
%\pagenumbering{arabic}
%Seitennummerierung in römischen Zahlen
\pagenumbering{Roman}
\setcounter{page}{0}


%Einbinden der Titelseite
%!TEX root = ../montravail.tex
\begin{titlepage}
\vspace*{-20ex}
\includegraphics[width=\textwidth]{grafiken/Linie}\\[2ex]
{\sffamily 
Fakultät für\\
Informations-, Medien-\\
und Elektrotechnik\\[10ex]
%\vspace*{-25cm}
\includegraphics[width=3cm]{grafiken/thk}\\
\begin{center}
 \Huge  Bachelor- oder Masterarbeit\\[4ex]
\end{center}
\Large
\textbf{Thema: Erstellen einer LaTeX Vorlage}
\large 
\begin{tabbing}
Matrikelnummer: \hspace*{2ex}\= \=\kill
Name: \> \textit{Name}\\
Anschrift:\> \textit{Anschrift}\\
Matrikelnummer:\> \textit{Matrikelnummer}	\\	
Studiengang:\> \textit{Studiengang}\\
\end{tabbing}
\begin{tabbing}
Ggf. externer Prüfer:\hspace*{2ex}\= \=\kill
Erstprüfer:\>\textit{Erstprüfer}\\			
Zweitprüfer:\>\textit{Zweitprüfer}\\
Ggf. externer Prüfer:\>\textit{Ggf. externer Prüfer} \\
\end{tabbing}
Anfertigungszeitraum:\\
Fertigstellung/Abgabedatum:\\[5ex]
%
Ich erkläre an Eides statt, dass ich die vorgelegte Abschlussarbeit selbständig 
und ohne fremde Hilfe verfasst, andere als die angegebenen Quellen und 
Hilfsmittel nicht benutzt und die den benutzten Quellen wörtlich oder inhaltlich 
entnommenen Stellen als solche kenntlich gemacht habe.\\[6ex]
%
Ort, Datum, Unterschrift: \hrulefill
}
\normalsize
\end{titlepage}

%%% Local Variables: 
%%% mode: latex
%%% TeX-master: 
%%% End:
%Zeilenabstand 1,5 fach
%\onehalfspacing
%Einbinden des Abstracts bzw. der Kurzfassung
%!TEX root = ../montravail.tex

 
\chapter*{Abstract} 
\lhead[ \leftmark   ]{\textbf{Abstract}}
\addcontentsline{toc}{chapter}{\large{Abstract}}

Hier kommt die Kurzfassung hin


\chapter*{Abstract}

und das Abstract wenn ben�tigt

Anleitung:

Motivation des Textes: 
worin liegt die Bedeutung der entsprechenden Forschung, warum sollte der l�ngere Text gelesen werden?

%Somit erfolgt eine einfache Aufz�hlung
\begin{itemize}
	\item Fragestellung: welche Fragestellung(en) versucht der Text zu beantworten, was ist der
Umfang der Forschung, was sind die zentralen Argumente und Behauptungen?
	\item Methodologie: welche Methoden/Zug�nge nutzt der Autor/die Autorin, auf welche
empirische Basis st�tzt sich der Text?
	\item Methodologie: welche Methoden/Zug�nge nutzt der Autor/die Autorin, auf welche
empirische Basis st�tzt sich der Text?
	\item Ergebnisse: zu welchen Ergebnissen kam die Forschung, was sind die zentralen
Schlussfolgerungen des Textes?
	\item Implikationen: welche Schlussfolgerungen ergeben sich aus dem Text f�r die Forschung,
was f�gt der Text unserem Wissen �ber das Thema hinzu?
\end{itemize}



%Zeilenabstand für das Inhaltsverzeichnis 1 fach
%\singlespacing
%Einbinden des Inhaltsverzeichnis
%!TEX root = ../montravail.tex
\lhead[ \leftmark   ]{\textbf{Contents}}
\tableofcontents

%Zeilenabstand für den Hauptteil ist 1,5 fach
%\singlespacing
%!TEX root = ../montravail.tex

  %Erzeugt ein Abbildungsverzeichnis
	%F�gt die Zeile "Abbildungsverzeichnis" als Chapter ins Inhaltsverzeichnis ein
	\addcontentsline{toc}{chapter}{List of figures}
	\lhead[ \leftmark   ]{\textbf{List of figures}}
	%Erzeugt ein Abbildungsverzeichnis
	%Schreibt Abb.: vor die Abbildung
	\renewcommand{\cftfigpresnum}{Fig. }
	\renewcommand{\cftfigaftersnum}{:}
	\setlength{\cftfignumwidth}{2cm}
	\setlength{\cftfigindent}{0cm}
	\listoffigures
	%mit dieser Zeile wird das entsprechende Verzeichnis nach links geschrieben
	\lhead[ \leftmark   ]{\textbf{List of figures}}
	
	\newpage
	%F�gt die Zeile "Tabellenverzeichnis" als Chapter ins Inhaltsverzeichnis ein
	\addcontentsline{toc}{chapter}{List of tables}
	%Erzeugt ein Abk�rzungsverzeichnis	
	\newpage
	%Schreibt Tab.: vor die Abbildung
	\renewcommand{\cfttabpresnum}{Tab. }
	\renewcommand{\cfttabaftersnum}{:}
	\setlength{\cfttabnumwidth}{2cm}
	\setlength{\cfttabindent}{0cm}
	%mit dieser Zeile wird das entsprechende Verzeichnis nach links geschrieben
	\lhead[ \leftmark   ]{\textbf{List of tables}}	
	\listoftables
		
	\newpage
	%F�gt die Zeile "Glossar" als Chapter ins Inhaltsverzeichnis ein
	\addcontentsline{toc}{chapter}{List of abbreviations}
	\newpage
	%mit dieser Zeile wird das entsprechende Verzeichnis nach links geschrieben
	\lhead[ \leftmark   ]{\textbf{List of abbreviations}}
	\printnomenclature


%	\newpage
%	\chapter*{Vorwort}
%	\addcontentsline{toc}{chapter}{Vorwort}
%	\lhead[ \leftmark   ]{\textbf{Vorwort}}
%	
	\newpage
%die Aufgabenstellung l�sst sich auch als eigenes Kapitel schreiben/Speichern
	\chapter*{Topic}
	\addcontentsline{toc}{chapter}{Topic}
	%mit dieser Zeile wird das entsprechende Verzeichnis nach links geschrieben
	\lhead[ \leftmark   ]{\textbf{Topic}}
	
	Die Idee an einem Unterst�tzungs-Roboter f�r gehbehinderte Menschen zu forschen, entstand durch das medial wachsende
	Interesse an Laufrobotern. Diese sind bereits im Entertainmentbereich z. B. bei Spielzeugen g�ngig, werden aber inzwischen
	auch\\
	vermehrt f�r Milit�rzwecke eingesetzt....
%Zeilenabstand für den Hauptteil ist 1,5 fach
%\onehalfspacing

%=== Hauptteil =======================================================
%Seitennummerierung des Hauptteils
\mainmatter
%Nummerierung beginnt beim Hauptteil ab Seite 6 (muss angepasst werden)
\pagenumbering{arabic}
\setcounter{page}{1}
	%Die Zähler für Tabellen und Abbildungen werden zurückgesetzt, damit

%Einbinden eines Vorwortes
%!TEX root = ../montravail.tex

\chapter*{Preface}
\addcontentsline{toc}{chapter}{Preface}
\lhead[ \leftmark   ]{\textbf{Preface}}

Das Vorwort ist optional.

%Manchmal so wie hier Grau umrandet, ich hatte kein Vorwort folglich ist es bei mir auskommentiert
%\noindent{}\fcolorbox{black}{sourcegray}{\parbox{\textwidth}{
%	Diese Diplomarbeit entstand mit freundlicher Unterstützung der Fachhochschule Köln
%	Campus Gummersbach, welche es ermöglichte Forschungen auf diesem Gebiet durchzuführen und somit das Projekt Transmover zu
%	zu initialisieren und vorzustellen. Für das entgegengebrachte Vertrauen
%	und Unterstützung gilt mein Dank Herrn Professor Dr. Hartmut Bärwolff und
%	Herrn Professor Dr. Holger Günther%}}
%	
%Abstand zwischen den Blöcken
%	\vspace{0.75cm}
%	
%	%\noindent{}\fcolorbox{black}{sourcegray}{\parbox{\textwidth}{
%	Des weiteren möchte ich meinen Eltern für Ihre Unterstützung im Studium
%	danken und Ihnen diese Arbeit widmen. Für angeregte Diskussionen und fachliche Kommentare, gilt mein Dank auch meinen
%	Professoren und Kommilitonen der Fachhochschule Köln Campus Gummersbach die diese Arbeit somit unterstützt haben.}}
	 	%in jedem Kapitel die Nummerierung neu beginnt
	
	\setcounter{chapter}{0}
	\setcounter{section}{1}
	\setcounter{table}{1}
	\setcounter{figure}{1}
	
	%Einbinden einer Einleitung
	\chapter*{Introduction}
\addcontentsline{toc}{chapter}{Introduction}
\lhead[ \leftmark   ]{\textbf{Introduction}}
%!Das Vorwort ist optional.\\ \\

%Manchmal Grau umrandet dies macht man so ich hatte kein Vorwort folglich ist es bei mir auskommentiert
%\noindent{}\fcolorbox{black}{sourcegray}{\parbox{\textwidth}{
%Die Schicksale auf unserer Erde sind teilweise so verschieden und zahlreich, das es kaum möglich sein wird den meisten Menschen Ihr Leben auf irgendeine Weise zu erleichtern oder sogar zu verbessern. 

%dies ist eine Mathematische Formel 
\begin{quotation}
\textit{\enquote{$A\rho\chi\eta\ \ \eta\mu\iota\sigma\upsilon\ \ \pi\alpha\nu\tau o \varsigma$ \\Der Anfang ist die Haelfte des Ganzen.}} (Vgl.\cite{111})
\end{quotation} 

%so wird eine korrekte Fußnote erstellt manche Professoren möchten hier auch Lieteraturverweise sehen
Roboter\footnote{Der Begriff Roboter (tschechisch: robot) wurde von Josef und Karel Capek Anfang des 20. Jahrhunderts durch die Science-Fiction-Literatur geprägt. Der Ursprung liegt im slawischen Wort robota, welches mit Arbeit, Fronarbeit oder Zwangsarbeit übersetzt werden kann. 1921 beschrieb Karel Capek in seinem Theaterstück R.U.R. in Tanks gezüchtete menschenähnliche künstliche Arbeiter. Mit seinem Werk greift Capek das klassische Motiv des Golems auf. Heute würde man Capeks Kunstgeschöpfe als Androiden bezeichnen. Vor der Prägung dieses Begriffes wurden Roboter zum Beispiel in den Werken von Stanislaw Lem als Automaten oder Halbautomaten bezeichnet.}

%ß wird in dieser Arbeit wie folgt gemacht
Bei Rädern und Ketten ist dies hingegen nicht der Fall, Sie brauchen gro\ss flächige Stützpunkte. 

%Zitate werden genormt so gemacht
Das bedeutet, dass letztere Arten der Fortbewegung einen ununterbrochenen Kontakt zum Boden benötigen. (Vgl.\cite{116})\\
%Allerdings fehlt hier noch die Seitenangabe "{Seite 4, ff}" das verlangen nicht alle Professoren so wie meiner 

%für den Index verwendete Wörter werden so angegeben
Dadurch wurden die im Verlauf der Evolution optimierten Konstruktionen von Beinen, das Zusammenspiel von Sensorik\index{Sensorik} und Aktorik\index{Aktorik} und die Steuerung von Gehbewegungen weitestgehend analysiert und dienten somit dem besseren Verständnis der Laufmotorik\footnote{Die Laufmotorik beinhaltet die Bewegungsfunktion und deren Lehre, die Fähigkeit des Körpers sich kontrolliert zu bewegen, die Gesamtheit der vom zentralen Nervensystem kontrollierten Bewegungen des Körpers im Gegensatz zu den unwillkürlichen Reflexen des Körpers und die Unterscheidung in Grob- und Feinmotorik}\index{Laufmotorik}.\\\\ 

An dieser Stelle sei darauf hingewiesen, dass dieses Forschungsgebiet einen stark interdisziplinären Charakter besitzt...


%Falls eine neue Seite erzwungen werden soll macht man dies so
\newpage
%um den Text Fett zu schreiben wir dieser Befehl verwendet
\textbf{Kapitelübersicht}\\\\  	%in jedem Kapitel die Nummerierung neu beginnt
	
	\setcounter{chapter}{0}
	\setcounter{section}{1}
	\setcounter{table}{1}
	\setcounter{figure}{1}
	
%Einbinden des ersten Kapitels
%!TEX root = ../Bachelorarbeit.tex

\chapter{Grundlagen}
%\addcontentsline{toc}{chapter}{Grundlagen}
\lhead[ \leftmark   ]{\textbf{Grundlagen}}

%erstes Unterkapitel
\section{Erste �berschrift}

Hier schreiben
%n�chstes unterkapitel ebene 2
\subsection{Zweite �berschrift}
Hier schreiben

%unterkapitel Ebene 3 nicht im Inhaltsverzeichnis aufgef�hrt! 
\subsubsection{Dritte �berschrift}
\textit{hier kann Kursiv geschrieben werden}

%zweites Unterkapitel
\section{Erste �berschrift}

%n�chstes unterkapitel ebene 2
\subsection{Zweite �berschrift}

%Abbildungsbeispiel
%Abbildungsquelle immer! angeben es sei den selber gemacht!!!!
Text....(siehe Abb. 1.1)

\begin{figure}[!ht]
	%mitte der Seite
	\centering
		%[nat�rliche Breite in Pixeln, nat�rliche H�he in Pixeln, Abh�ngigkeit von der Textbreite]
		\includegraphics[natwidth=1200pt, natheight=349pt, width=0.4\textwidth]{grafiken/Robotpeintre.PNG}
	\caption[Aufbau allgemein]{Aufbau und Komponenten von Robotern}
	\label{fig:Aufbau und Komponenten von Robotern}
\end{figure}

%unterkapitel Ebene 3 nicht im Inhaltsverzeichnis aufgef�hrt!
\subsubsection{Dritte �berschrift}

%Aufz�hlung
\begin{itemize}
	\item Die Bewegungsform der Achsen
	\item Anzahl und Anordnung der Achsen
	\item Die Formen des Arbeitsraums
\end{itemize}
    
... Arm zu strecken. (Vgl.\cite{112}) 
%n�chstes unterkapitel ebene 2
\subsection{Zweite �berschrift}
%unterkapitel Ebene 3 nicht im Inhaltsverzeichnis aufgef�hrt!
\subsubsection{Dritte �berschrift}
Hier Text einf�gen\index{einf�gen}

%� ohne weiterf�hrung des Wortes
Roboterfu\ss \ befindet

homogene\\ 4 x 4  Matrix:

\begin{equation} 
T = \begin{pmatrix}
     Ax&Ay&Az&0\\Bx&By&Bz&0\\Cx&Cy&Cz&0\\Px&Py&Pz&1
     \end{pmatrix}
\end{equation}


\begin{equation}(\theta, d, a, \alpha)\end{equation}
 
verschiedene Matrizen:

\begin{equation}
T=\begin{pmatrix}\cos\theta & -\sin\theta \cos\alpha & \sin\theta \sin\alpha & \arccos\theta\\ \sin\theta & \cos\theta \cos\alpha & -\cos\theta \sin\alpha & \arcsin\theta\\ 0 & \sin\alpha & \cos\alpha & d\\ 0 & 0 & 0 & 1\end{pmatrix}
\end{equation}

\begin{equation}
^{n - 1}T_n
  = \begin{pmatrix}
    \cos\theta_n & -\sin\theta_n \cos\alpha_n & \sin\theta_n \sin\alpha_n & a_n \cos\theta_n \\
    \sin\theta_n & \cos\theta_n \cos\alpha_n & -\cos\theta_n \sin\alpha_n & a_n \sin\theta_n \\
    0 & \sin\alpha_n & \cos\alpha_n & d_n \\
    0 & 0 & 0 & 1
  \end{pmatrix}.
\end{equation}


\begin{equation}
T=T_1T_2T_3T_4T_5T_{tcp}
\end{equation}


%Beispiel f�r eine Tabelle
Text...(Siehe Tab. 1.1)
\\\\
\textbf{Titel:}\\%Der Titel bei Tabellen muss immer �ber der Tabelle stehen Statuten FH-Koeln! bei Abbildungen drunter! 
\begin{table}[ht]
\centering
\caption[Titel]{Titel}	
	 \begin{tabular}{|c|p{11cm}|}
			\hline
			\rowcolor{sourcegray}
			\textbf{�berschrift 1} & \textbf{�berschrift 2}\\
			\hline
			Text & Text \newline 
						 Text\\
			\hline
			Text & Text \newline 
						 Text\\
			\hline
		\end{tabular}
	\vspace{1.0em}
	%\caption[Mobilit�tsgrade]{Gliederung der Mobilit�tsgrade}
	\label{tab:Titel}
\end{table}

\newpage
%Beispiel f�r zwei Bilder nebeneinander
\begin{figure}[!ht]
\centering 
\begin{minipage}[hbt]{7cm}
	\includegraphics[width=7cm]{grafiken/HONDA_ASIMO.jpg}
\end{minipage}
\begin{minipage}[hbt]{6.7cm}
	\includegraphics[width=6.7cm]{grafiken/Toyota_Robot_at_Toyota_Kaikan.jpg}
	\end{minipage}
	\caption[Humanoide Roboter]{Humanoide Roboter Links der Aibo von Honda rechts der Roboter von Toyota}
	\label{Humanoide Roboter}
\end{figure}

Bild-Quellen:(\url{http://de.wikipedia.org/wiki/Humanoider_Roboter})\\Sichtung: 17.09.2010\\\\

In Abb. 1.2 sind...Text\\\\
 
%�bersichtshalber besser Begriffe betonen
Kraft \textbf{F}, der Masse \textbf{m} und der Beschleunigung \textbf{a} kann mit der daraus resultieren Formel die Kraft, die wirkt, berechnet werden: \begin{equation}F = m * a \end{equation} Folglich ist die Kraft das Produkt von Masse und Beschleunigung.  

%Formeln 
SI-Einheit der Kraft:
\begin{equation}
[F] = kg * \frac{m}{s^{2}} = Newton (N)
\end{equation}

\begin{equation}M = F * l = F * r * \sin\alpha \end{equation}
 
SI-Einheit des Drehmomentes: \begin{equation}[M] = Newtonmeter (N * m)\end{equation}
  

	%Die Zähler für Tabellen und Abbildungen werden zurückgesetzt, damit
	%in jedem Kapitel die Nummerierung neu beginnt
	\setcounter{table}{1}
	\setcounter{figure}{1}
 %Einbinden des ersten Kapitels a
%!TEX root = ../Montravail.tex

\chapter{Language Syntax}
%\addcontentsline{toc}{chapter}{Grundlagen}
\lhead[ \leftmark   ]{\textbf{Language Syntax}}

%erstes Unterkapitel
\section{express structural concepts}

\textcolor[rgb]{1,0,0}{Why do we need a description?
? Designers: to communicate ideas
? Implementors: to build a conforming compiler
? Programmers: how do I write a while loop in language X?}

%n�chstes unterkapitel ebene 2
\subsection{features to describe}
? syntax: legal sentences
? semantics
? static: checked at compile time
? dynamic: run time behaviour\\\\
a simple, compact and unambiguous notation:
? We need a formal description rather than an informal (natural) language
description.
%n�chstes unterkapitel ebene 2
\subsection{Zweite �berschrift}
the ASCI character version\\\\
lexems identifiers and strings
numbers are built-in primitive symbols

%unterkapitel Ebene 3 nicht im Inhaltsverzeichnis aufgef�hrt! 
\subsubsection{the language evolution}
\textit{hier kann Kursiv geschrieben werden}
xy were sentences and equals was a verb
xy are nouns and equals is an operation
Just as the noun yx names the ...
Many people think of all mathematical expressions as sentences, reading 1+1
as Add 1 and 1. They think of 1 + 1 = 2 as asserting
%zweites Unterkapitel
\section{Well-writen}
how to write a module?
%n�chstes unterkapitel ebene 2
\subsection{qualities}
written in a declarative style that has the goal of making
%Abbildungsbeispiel
%Abbildungsquelle immer! angeben es sei den selber gemacht!!!!
Text....(siehe Abb. 1.1)

\begin{figure}[!ht]
	%mitte der Seite
	\centering
		%[nat�rliche Breite in Pixeln, nat�rliche H�he in Pixeln, Abh�ngigkeit von der Textbreite]
		\includegraphics[natwidth=1200pt, natheight=349pt, width=0.4\textwidth]{grafiken/Robotpeintre.PNG}
	\caption[Aufbau allgemein]{Aufbau und Komponenten von Robotern}
	\label{fig:Aufbau und Komponenten von Robotern}
\end{figure}

\subsection{TLA+}
%unterkapitel Ebene 3 nicht im Inhaltsverzeichnis aufgef�hrt!
\subsubsection{Propositional logic is the study of simple operations on Booleans (truth values)}
all the opssilies tha can occur , as probabilitee
all the cases that can be defined with the semantic
%Aufz�hlung
\begin{itemize}
	\item introduce an extra level of indirection between your domain abstraction and the lan
	guage of implementation.
	\item  extensible abstractions 
	\item Die Formen des Arbeitsraums
\end{itemize}
    
... extensible abstractions. (Vgl.\cite{112})
as a manager translates the sentence into a set of separate sentences, checks the ones that are trivially true, and sends the others to one
or more other ...\\\\
manager remembers if it has already .. an .. \\\\
This would be inelegant but feasible\\\\
 being completely rigorous
and using the notation of dsl2ttcn3 , you would have to write G[1] and G[2] instead
of N and E . (If you were restricted to ordinary dsl notation, you would
have no way to be rigorous.)\\\\
%n�chstes unterkapitel ebene 2
\subsection{implementation inheritance}
%unterkapitel Ebene 3 nicht im Inhaltsverzeichnis aufgef�hrt!
\subsubsection{Dritte �berschrift}
dsl in action\index{A.2.2 Subtyping to prevent implementation leak}
\textcolor[rgb]{1,0,0}{subclasses become unnecessarily coupled to the implementation of your base class!
how to distill an abstraction out of its nonessential details
minimize accidental complexity in your abstractions}
%� ohne weiterf�hrung des Wortes
Roboterfu\ss \ befindet

homogene\\ 4 x 4  Matrix:

\begin{equation} 
T = \begin{pmatrix}
     Ax&Ay&Az&0\\Bx&By&Bz&0\\Cx&Cy&Cz&0\\Px&Py&Pz&1
     \end{pmatrix}
\end{equation}


\begin{equation}(\theta, d, a, \alpha)\end{equation}
 
verschiedene Matrizen:

\begin{equation}
T=\begin{pmatrix}\cos\theta & -\sin\theta \cos\alpha & \sin\theta \sin\alpha & \arccos\theta\\ \sin\theta & \cos\theta \cos\alpha & -\cos\theta \sin\alpha & \arcsin\theta\\ 0 & \sin\alpha & \cos\alpha & d\\ 0 & 0 & 0 & 1\end{pmatrix}
\end{equation}

\begin{equation}
^{n - 1}T_n
  = \begin{pmatrix}
    \cos\theta_n & -\sin\theta_n \cos\alpha_n & \sin\theta_n \sin\alpha_n & a_n \cos\theta_n \\
    \sin\theta_n & \cos\theta_n \cos\alpha_n & -\cos\theta_n \sin\alpha_n & a_n \sin\theta_n \\
    0 & \sin\alpha_n & \cos\alpha_n & d_n \\
    0 & 0 & 0 & 1
  \end{pmatrix}.
\end{equation}


\begin{equation}
T=T_1T_2T_3T_4T_5T_{tcp}
\end{equation}


%Beispiel f�r eine Tabelle
Text...(Siehe Tab. 1.1)
\\\\
\textbf{Titel:}\\%Der Titel bei Tabellen muss immer �ber der Tabelle stehen Statuten FH-Koeln! bei Abbildungen drunter! 
\begin{table}[ht]
\centering
\caption[Titel]{Titel}	
	 \begin{tabular}{|c|p{11cm}|}
			\hline
			\rowcolor{sourcegray}
			\textbf{�berschrift 1} & \textbf{�berschrift 2}\\
			\hline
			Text & Text \newline 
						 Text\\
			\hline
			Text & Text \newline 
						 Text\\
			\hline
		\end{tabular}
	\vspace{1.0em}
	%\caption[Mobilit�tsgrade]{Gliederung der Mobilit�tsgrade}
	\label{tab:Titel}
\end{table}

\newpage
%erstes Unterkapitel
\section{The process of writing a correct specification }

\textcolor[rgb]{1,0,0}{This is a procedure I use regularly when writing code that is at all tricky and that I want to be correct. But, getting this program correct was not important
enough to warrant spending that much time. ???}

%n�chstes unterkapitel ebene 2
\subsection{features to describe}
most obviously correct specifications could not be executed efficiently
The thing to do in such a case is to write two versions of some definitions:
one that is simpler and more obviously correct, and another that dsl can evaluate more efficiently.
%n�chstes unterkapitel ebene 2
\subsection{Sorting the elements of the specification}


We want to specify what it means to sort a set of elements. For simplicity, let?s
assume that each element is a record ? * with a key component, and the elements
are to be sorted in increasing order of their key value.
Sorting a set means arranging its elements in a list. A sorting of a set S is
therefore a list of elements of S that contains each element exactly once and is
sorted according to the elements? key values. To describe this precisely, we must
define:
? What a list of elements of S is.
? What it means for such a list to contain each element of S exactly once.
? What it means for the list to be sorted according to elements? key values.
\\\\
However,
if she defined a Turing machine T to be a 7-tuple, it would be impossible to
remember if its initial state was T [4] or T [5]. Programming languages solve this problem by introducing records.\\\\
the record\\\\
*Records in Programming Languages
A record is called a struct in the C programming language. In more modern
programming languages, an object is a record for which certain operations are
defined. In addition to using r .f to mean the value of field f of record r ,
these languages also use r .O(. . .) to mean the value obtained (and side effects
produced) by appying the operation O to the record r .
If you program in an object-oriented language, you may miss some of its fea-
tures when using TLA + . While those features are useful for writing programs,
they are not needed for writing specifications, and their inherent mathematical
complexity would make specifications that used them harder to understand.

%n�chstes unterkapitel ebene 2
\subsection{The Hierarchical Structure}

To write the spec, I had to decide how to represent the voters?
ballots. The obvious way to represent a single voter?s ballot is by a sequence
whose i th element is the name of the candidate the voter ranks number i . An
obvious way to represent all the votes is as a set of such sequences. However,
that?s not right because two voters can cast identical ballots, and there is no
concept of a set having ?two copies? of an element. Here are three reasonable
ways to represent the collection of votes:
? With a set whose elements are records or tuples with one component being
the ranking and the other identifying the voter (with a randomly chosen
identification if the vote is to be anonymous).
? With a bag (multiset) ? of rankings.
? With a sequence of rankings, arranged in an arbitrary order.
%n�chstes unterkapitel ebene 2
\subsection{System as a formula}
diferent specifications lead to the same Module definition
one standard interpretation

%unterkapitel Ebene 3 nicht im Inhaltsverzeichnis aufgef�hrt!
\subsubsection{Writing Structured specifications}
dsl in action\index{A.2.2 Subtyping to prevent implementation leak}
\textcolor[rgb]{1,0,0}{A module consists of module definition describing the module, followed by either:
Writing Structured optional module control part
? A non-module control that is a sequence of steps, ending with a qed step, each of which may (but need not) have a control part.
? A module control, which is either the keyword obvious, the keyword omitted, or a by ....}

after\\\\
conjunction... formula like tla+



\newpage
%Beispiel f�r zwei Bilder nebeneinander
\begin{figure}[!ht]
\centering 
\begin{minipage}[hbt]{7cm}
	\includegraphics[width=7cm]{grafiken/HONDA_ASIMO.jpg}
\end{minipage}
\begin{minipage}[hbt]{6.7cm}
	\includegraphics[width=6.7cm]{grafiken/Toyota_Robot_at_Toyota_Kaikan.jpg}
	\end{minipage}
	\caption[Humanoide Roboter]{Humanoide Roboter Links der Aibo von Honda rechts der Roboter von Toyota}
	\label{Humanoide Roboter}
\end{figure}

Bild-Quellen:(\url{http://de.wikipedia.org/wiki/Humanoider_Roboter})\\Sichtung: 17.09.2010\\\\

In Abb. 1.2 sind...Text\\\\
 
%�bersichtshalber besser Begriffe betonen
Kraft \textbf{F}, der Masse \textbf{m} und der Beschleunigung \textbf{a} kann mit der daraus resultieren Formel die Kraft, die wirkt, berechnet werden: \begin{equation}F = m * a \end{equation} Folglich ist die Kraft das Produkt von Masse und Beschleunigung.  

%Formeln 
SI-Einheit der Kraft:
\begin{equation}
[F] = kg * \frac{m}{s^{2}} = Newton (N)
\end{equation}

\begin{equation}M = F * l = F * r * \sin\alpha \end{equation}
 
SI-Einheit des Drehmomentes: \begin{equation}[M] = Newtonmeter (N * m)\end{equation}
  

	%Die Zähler für Tabellen und Abbildungen werden zurückgesetzt, damit
	%in jedem Kapitel die Nummerierung neu beginnt
	\setcounter{table}{1}
	\setcounter{figure}{1}
 %Einbinden des ersten Kapitels a
%!TEX root = ../Montravail.tex

\chapter{Role of abstractions
in domain modeling}
%\addcontentsline{toc}{chapter}{Grundlagen}
\lhead[ \leftmark   ]{\textbf{Role of abstractions
in domain modeling}}

%erstes Unterkapitel
\section{basis for description and analysis}

\textcolor[rgb]{1,0,0}{A DSL is nothing but a layer of abstraction over an underlying implementation model.
Unless you get it right, your domain model won?t be at the correct level of abstraction, and the linguistic representation of it in the DSL won?t be either.
well-engineered abstraction, do not contains nonessential details that don?t belong to the core concerns of the abstraction
design abstractions at the correct level to
ensure that your model speaks the language of the domain
abstraction needs to publish only the core essence
to its clients, which makes it minimal from the outside looking in.}

%n�chstes unterkapitel ebene 2
\subsection{Zweite �berschrift}
Every abstraction has a functionality to deliver to its clients
minimality of abstractions.
distillation of the abstraction
%n�chstes unterkapitel ebene 2
\subsection{Zweite �berschrift}
Extensibility and composability
Extensibility is possible only through composability
Engineering is about designing things in a modular way
%unterkapitel Ebene 3 nicht im Inhaltsverzeichnis aufgef�hrt! 
\subsubsection{Dritte �berschrift}
\textit{hier kann Kursiv geschrieben werden}
Extensibility and composability
Extensibility is possible only through composability
Engineering is about designing things in a modular way
Well-behaved abstractions can be composed to form higher-level abstractions
%zweites Unterkapitel
\section{Well-designed abstractions}

%n�chstes unterkapitel ebene 2
\subsection{qualities}
abstractions at the correct level to ensure that your model speaks the language of the domain. Only then will your code be as expressive as the language of the person who?s an expert in the domain.
%Abbildungsbeispiel
%Abbildungsquelle immer! angeben es sei den selber gemacht!!!!
Text....(siehe Abb. 1.1)

\begin{figure}[!ht]
	%mitte der Seite
	\centering
		%[nat�rliche Breite in Pixeln, nat�rliche H�he in Pixeln, Abh�ngigkeit von der Textbreite]
		\includegraphics[natwidth=1200pt, natheight=349pt, width=0.4\textwidth]{grafiken/Robotpeintre.PNG}
	\caption[Aufbau allgemein]{Aufbau und Komponenten von Robotern}
	\label{fig:Aufbau und Komponenten von Robotern}
\end{figure}

%unterkapitel Ebene 3 nicht im Inhaltsverzeichnis aufgef�hrt!
\subsubsection{how to design an abstraction}

%Aufz�hlung
\begin{itemize}
	\item introduce an extra level of indirection between your domain abstraction and the lan
	guage of implementation.
	\item  extensible abstractions 
	\item Die Formen des Arbeitsraums
\end{itemize}
    
... extensible abstractions. (Vgl.\cite{112})
difficult to extend java.util.Map using a true OO approach \\\\
Mixins let you do exactly what?s needed here \\\\
Scala offers mixin implementation in the form of traits.\\\\
in the final implementation, the trait xy is mixed in
dynamically during runtime object creation.\\\\
%n�chstes unterkapitel ebene 2
\subsection{implementation inheritance}
%unterkapitel Ebene 3 nicht im Inhaltsverzeichnis aufgef�hrt!
\subsubsection{Dritte �berschrift}
dsl in action\index{A.2.2 Subtyping to prevent implementation leak}
\textcolor[rgb]{1,0,0}{subclasses become unnecessarily coupled to the implementation of your base class!
how to distill an abstraction out of its nonessential details
minimize accidental complexity in your abstractions}
%� ohne weiterf�hrung des Wortes
Roboterfu\ss \ befindet

homogene\\ 4 x 4  Matrix:

\begin{equation} 
T = \begin{pmatrix}
     Ax&Ay&Az&0\\Bx&By&Bz&0\\Cx&Cy&Cz&0\\Px&Py&Pz&1
     \end{pmatrix}
\end{equation}


\begin{equation}(\theta, d, a, \alpha)\end{equation}
 
verschiedene Matrizen:

\begin{equation}
T=\begin{pmatrix}\cos\theta & -\sin\theta \cos\alpha & \sin\theta \sin\alpha & \arccos\theta\\ \sin\theta & \cos\theta \cos\alpha & -\cos\theta \sin\alpha & \arcsin\theta\\ 0 & \sin\alpha & \cos\alpha & d\\ 0 & 0 & 0 & 1\end{pmatrix}
\end{equation}

\begin{equation}
^{n - 1}T_n
  = \begin{pmatrix}
    \cos\theta_n & -\sin\theta_n \cos\alpha_n & \sin\theta_n \sin\alpha_n & a_n \cos\theta_n \\
    \sin\theta_n & \cos\theta_n \cos\alpha_n & -\cos\theta_n \sin\alpha_n & a_n \sin\theta_n \\
    0 & \sin\alpha_n & \cos\alpha_n & d_n \\
    0 & 0 & 0 & 1
  \end{pmatrix}.
\end{equation}


\begin{equation}
T=T_1T_2T_3T_4T_5T_{tcp}
\end{equation}


%Beispiel f�r eine Tabelle
Text...(Siehe Tab. 1.1)
\\\\
\textbf{Titel:}\\%Der Titel bei Tabellen muss immer �ber der Tabelle stehen Statuten FH-Koeln! bei Abbildungen drunter! 
\begin{table}[ht]
\centering
\caption[Titel]{Titel}	
	 \begin{tabular}{|c|p{11cm}|}
			\hline
			\rowcolor{sourcegray}
			\textbf{�berschrift 1} & \textbf{�berschrift 2}\\
			\hline
			Text & Text \newline 
						 Text\\
			\hline
			Text & Text \newline 
						 Text\\
			\hline
		\end{tabular}
	\vspace{1.0em}
	%\caption[Mobilit�tsgrade]{Gliederung der Mobilit�tsgrade}
	\label{tab:Titel}
\end{table}

\newpage
%Beispiel f�r zwei Bilder nebeneinander
\begin{figure}[!ht]
\centering 
\begin{minipage}[hbt]{7cm}
	\includegraphics[width=7cm]{grafiken/HONDA_ASIMO.jpg}
\end{minipage}
\begin{minipage}[hbt]{6.7cm}
	\includegraphics[width=6.7cm]{grafiken/Toyota_Robot_at_Toyota_Kaikan.jpg}
	\end{minipage}
	\caption[Humanoide Roboter]{Humanoide Roboter Links der Aibo von Honda rechts der Roboter von Toyota}
	\label{Humanoide Roboter}
\end{figure}

Bild-Quellen:(\url{http://de.wikipedia.org/wiki/Humanoider_Roboter})\\Sichtung: 17.09.2010\\\\

In Abb. 1.2 sind...Text\\\\
 
%�bersichtshalber besser Begriffe betonen
Kraft \textbf{F}, der Masse \textbf{m} und der Beschleunigung \textbf{a} kann mit der daraus resultieren Formel die Kraft, die wirkt, berechnet werden: \begin{equation}F = m * a \end{equation} Folglich ist die Kraft das Produkt von Masse und Beschleunigung.  

%Formeln 
SI-Einheit der Kraft:
\begin{equation}
[F] = kg * \frac{m}{s^{2}} = Newton (N)
\end{equation}

\begin{equation}M = F * l = F * r * \sin\alpha \end{equation}
 
SI-Einheit des Drehmomentes: \begin{equation}[M] = Newtonmeter (N * m)\end{equation}

%erstes Unterkapitel
\section{System verification and validation}

\textcolor[rgb]{1,0,0}{
\begin{center}
\textbf{das musss weg!}
\end{center}
Validation: are we building the right system ?\\
Verification: are we building the system right ?}




  

	%Die Zähler für Tabellen und Abbildungen werden zurückgesetzt, damit
	%in jedem Kapitel die Nummerierung neu beginnt
	\setcounter{table}{1}
	\setcounter{figure}{1}
	 
%!TEX root = ../Bachelorarbeit.tex

\chapter{Marktanalyse}
\lhead[ \leftmark   ]{\textbf{Marktanalyse}}
Aufbau siehe Kapitel 1

%Aufbau des Abk�rzungsverezeichnisses muss nicht hier stehen geht in jedem Kapitel
%======================================================================
%======================================================================
%Eintrag ins Abk�rzungsverzeichnis
\nomenclature{IBK}{Initialer Bodenkontakt}
\nomenclature{BA}{Belastungsantwort}
\nomenclature{BIC}{Business Innovation Center}
\nomenclature{USB}{Universal Serial Bus}
\nomenclature{KOS}{Koordinatensysteme}
\nomenclature{TCP}{Tool Center Point}
\nomenclature{SIG}{Special Interest Group}
\nomenclature{WPAN}{Wireless Personal Area Network}
\nomenclature{SRD}{Short Range Devices}
\nomenclature{ISM-Band}{Industrial, Scientific and Medical Band}
\nomenclature{WLAN}{Wireless Local Area Network}
\nomenclature{EDR}{Enhanced Data Rate}
\nomenclature{SPP}{Serial Port Profile}
\nomenclature{LAN}{Local Area Network}
\nomenclature{MKS}{Mehrk�rpersystem}
\nomenclature{DFG}{Deutsche Forschungsgemeinschaft}
\nomenclature{JARA}{Japan Robot Association}
\nomenclature{RIA}{Robot Institut of America}
\nomenclature{VDI}{Verein Deutscher Ingenieure}
\nomenclature{ISO}{International Organization for Standardization}
\nomenclature{IEEE}{Institute of Electrical and Electronics Engineers}
\nomenclature{MPL}{Mozilla Public License}
\nomenclature{Pose}{Position und Orientierung}
\nomenclature{SOC}{System on a Chip}
\nomenclature{CAN}{Controller Area Network}
\nomenclature{LIN}{Local Interconnect Network}
\nomenclature{MCU}{Mikrocontroller Unit}
\nomenclature{ISC}{Inter-Integrated Circuit}
\nomenclature{SPI}{Serial Peripheral Interface}
\nomenclature{LCD}{Liquid Crystal Display}
\nomenclature{PWM}{Pulse Width Modulation}
\nomenclature{ROM}{Read only Memory}
\nomenclature{EPROM}{Erasable Programmable Read Only Memory}
\nomenclature{UV}{Ultra Violet}
\nomenclature{OTP}{One Time Programmable}
\nomenclature{RAM}{Random Acess Memory}
\nomenclature{ARM7}{Advanced RISC Machines}
\nomenclature{PDA}{Personal Digital Assistant}
\nomenclature{CSR}{Core Serial Protocol}
\nomenclature{OS}{Oberschenkel}
\nomenclature{MSw}{Mittlere Schwungphase}
\nomenclature{TD}{Touch Down}
\nomenclature{MSt}{Mittlere Standphase}
\nomenclature{TSt}{Terminale Standphase}
\nomenclature{VSw}{Vor-Schwungphase}
\nomenclature{TSw}{Terminale Schwungphase}
\nomenclature{TO}{Take Off}
\nomenclature{ISw}{Initiale Schwungphase}
\nomenclature{SG}{Sprunggelenk}
\nomenclature{KG}{Kniegelenk}
\nomenclature{KI}{K�nstliche Intelligenz}
\nomenclature{DOF}{Degrees of freedom}




\newpage
%\addcontentsline{toc}{chapter}{Ergebnisse}
\lhead[ \leftmark   ]{\textbf{Ergebnisse}}
\section{dfhdhgj}
Aufbau siehe Kapitel 1 

\section{thsgfnfgnj}
Text

\section{gfhfgjdrtu}
Text

	
	%Die Zähler für Tabellen und Abbildungen werden zurückgesetzt, damit
	%in jedem Kapitel die Nummerierung neu beginnt
	\setcounter{table}{1}
	\setcounter{figure}{1}
	%Einbinden des zweiten Kapitels
%!TEX root = ../Bachelorarbeit.tex

\chapter{Entwicklung des Modells}
\lhead[ \leftmark   ]{\textbf{Modell Realisierung}}
Aufbau siehe Kapitel 1 

\section{Erstes Unterkapitel}
%Beispiel f�r nummerierte Aufz�hlung
\begin{enumerate}
	\item Text\\
	Text
	\item Text\\
	Text
	\item Text\\
	Text
\end{enumerate} 

\newpage
%Eingaben von Source Code
Zuerst erfolgt zum besseren Verst�ndnis die Deklarierung der verwendeten Variablen\\
\begin{lstlisting}[basicstyle=\small,numbers=left,numberstyle=\footnotesize\small,backgroundcolor=\color{sourcegray}]
// die Motoren des rechten Beines werden wie in dem bisher 
// verwendeten Schema zugewiesen 
// H�fte (C), Knie (B) und Fussgelenk (A)
public static Motor motorHip = new Motor(MotorPort.C);
public static Motor motorKnee = new Motor(MotorPort.B);
public static Motor motorAnkle = new Motor(MotorPort.A);
public static boolean sensorReached1 = false;

// der Ultraschallsensor wird an Port S1 erwartet
public static UltrasonicSensor sonicSensor1 = 
              new UltrasonicSensor(SensorPort.S1);

//Der TouchSensor wird an Port S2 erwartet und wird zum iterieren
// durch einen Bewegungsablauf verwendet
public static TouchSensor touchSensor = 
              new TouchSensor(SensorPort.S2);
	
\end{lstlisting}

%\noindent{}Beispiel f�r eine nummerierte Aufz�hlung:
\begin{enumerate}
	\item Item 1
	\item Item 2	
	\item Item 3
\end{enumerate}

%\noindent{}Beispiel f�r eine unnummerierte Aufz�hlung mit neuem Symbol:
\begin{itemize}
\renewcommand{\labelitemi}{$\rightarrow$}
	\item Item 1
	\item Item 2
	\item Item 3
\end{itemize}

%Referenz zu Grafik \ref{fig:showcase} in Kapitel \ref{sec:section}.



	%Die Zähler für Tabellen und Abbildungen werden zurückgesetzt, damit
	%in jedem Kapitel die Nummerierung neu beginnt
	\setcounter{table}{1}
	\setcounter{figure}{1}
	%Einbinden des zweiten Kapitels
%!TEX root = ../Montravail.tex

\chapter{algebraic design}
\addcontentsline{toc}{chapter}{xyhas to be defined}
\lhead[ \leftmark   ]{\textbf{xyhas to be defined}}
%!Das Vorwort ist optional.\\ \\

%Manchmal Grau umrandet dies macht man so ich hatte kein Vorwort folglich ist es bei mir auskommentiert
%\noindent{}\fcolorbox{black}{sourcegray}{\parbox{\textwidth}{
%Die Schicksale auf unserer Erde sind teilweise so verschieden und zahlreich, das es kaum m�glich sein wird den meisten Menschen Ihr Leben auf irgendeine Weise zu erleichtern oder sogar zu verbessern. 

%dies ist eine Mathematische Formel 
\begin{quotation}
\textit{\enquote{$A\rho\chi\eta\ \ \eta\mu\iota\sigma\upsilon\ \ \pi\alpha\nu\tau o \varsigma$ \\Der Anfang ist die H�lfte des Ganzen.}} (Vgl.\cite{111})
\end{quotation} 

%so wird eine korrekte Fu�note erstellt manche Professoren m�chten hier auch Lieteraturverweise sehen
Roboter\footnote{Der Begriff Roboter (tschechisch: robot) wurde von Josef und Karel Capek Anfang des 20. Jahrhunderts durch die Science-Fiction-Literatur gepr�gt. Der Ursprung liegt im slawischen Wort robota, welches mit Arbeit, Fronarbeit oder Zwangsarbeit �bersetzt werden kann. 1921 beschrieb Karel Capek in seinem Theaterst�ck R.U.R. in Tanks gez�chtete menschen�hnliche k�nstliche Arbeiter. Mit seinem Werk greift Capek das klassische Motiv des Golems auf. Heute w�rde man Capeks Kunstgesch�pfe als Androiden bezeichnen. Vor der Pr�gung dieses Begriffes wurden Roboter zum Beispiel in den Werken von Stanislaw Lem als Automaten oder Halbautomaten bezeichnet.}

\section{Parser combinators vs. parser generators}
%� wird in dieser Arbeit wie folgt gemacht
introduce a design approach called algebraic design \\\\
Parser combinators vs. parser generators

%Zitate werden genormt so gemacht
Das bedeutet, dass letztere Arten der Fortbewegung einen ununterbrochenen Kontakt zum Boden ben�tigen. (Vgl.\cite{116})\\
%Allerdings fehlt hier noch die Seitenangabe "{Seite 4, ff}" das verlangen nicht alle Professoren so wie meiner 

%f�r den Index verwendete W�rter werden so angegeben
Dadurch wurden die im Verlauf der Evolution optimierten Konstruktionen von Beinen, das Zusammenspiel von Sensorik\index{Sensorik} und Aktorik\index{Aktorik} und die Steuerung von Gehbewegungen weitestgehend analysiert und dienten somit dem besseren Verst�ndnis der Laufmotorik\footnote{Die Laufmotorik beinhaltet die Bewegungsfunktion und deren Lehre, die F�higkeit des K�rpers sich kontrolliert zu bewegen, die Gesamtheit der vom zentralen Nervensystem kontrollierten Bewegungen des K�rpers im Gegensatz zu den unwillk�rlichen Reflexen des K�rpers und die Unterscheidung in Grob- und Feinmotorik}\index{Laufmotorik}.\\\\ 

An dieser Stelle sei darauf hingewiesen, dass dieses Forschungsgebiet einen stark interdisziplin�ren Charakter besitzt...


%Falls eine neue Seite erzwungen werden soll macht man dies so
\newpage
%um den Text Fett zu schreiben wir dieser Befehl verwendet
\textbf{Kapitel�bersicht}\\\\ 
%Falls eine neue Seite erzwungen werden soll macht man dies so
\newpage
\section{SBT build system}
build and run your Scala code
using sbt, the build tool for Scala, and/or an IDE like IntelliJ or Eclipse. See the book's source code repo on GitHub for more information on getting set up with sbt.
Sbt is very smart about ensuring only the minimum number of files are recompiled when changes are made. It also has a number of other nice features which we won't
discuss here.


sbt uses a small number of concepts to support **flexible and powerful build definitions**. There are not that many concepts, but sbt is not exactly like other build systems and there are details you�will�stumble on if you haven?t read the documentation
	
	%Die Zähler für Tabellen und Abbildungen werden zurückgesetzt, damit
	%in jedem Kapitel die Nummerierung neu beginnt
	\setcounter{table}{1}
	\setcounter{figure}{1}
	%Einbinden des zweiten Kapitels
%!TEX root = ../Bachelorarbeit.tex

\chapter{Ergebnisse}
%\addcontentsline{toc}{chapter}{Ergebnisse}
\lhead[ \leftmark   ]{\textbf{Ergebnisse}}
\section{Pr�sentation}
Aufbau siehe Kapitel 1 

\section{Schwierigkeiten}
Text

\section{Auswertungen}
Text

	%Die Zähler für Tabellen und Abbildungen werden zurückgesetzt, damit
	%in jedem Kapitel die Nummerierung neu beginnt
	\setcounter{table}{1}
	\setcounter{figure}{1}
	%Einbinden des zweiten Kapbitels

\chapter{Zusammenfassung}
%\addcontentsline{toc}{chapter}{Zusammenfassung und Ausblick}
\lhead[ \leftmark   ]{\textbf{Zusammenfassung und Ausblick}}
\section {Zusammenfassung}
Text

\section{Ausblick auf zuk�nftige Arbeiten}
Text

\section{Schlusswort}
Text

%Zeilenabstand 1 fach für die Verzeichnisse
%\halfspacing
\singlespacing
%Einbindne der Verzeichnisse
\include{kapitel/verzeichnisse}

%=== Schlussteil =====================================================
%%Seitennummerierung für den Anhang
%\backmatter
%Seitennummerierung in römischen Zahlen
%\pagenumbering{Roman}

	%ändert den Stil des Literaturverzeichnisses List of references
  \lhead[ \leftmark   ]{\textbf{Bibliography}}
  \bibliographystyle{geralpha}
	%Erzeugt das Literaturverzeichnis anhand der Datei "`literatur.bib"'
	\bibliography{literatur}
	%Fügt die Zeile "`Literaturverzeichnis"' als Chapter ins Inhaltsverzeichnis ein
	\addcontentsline{toc}{chapter}{\large{Bibliography}}

	
%Fügt die Zeile "`Anhang"' als Part ins Inhaltsverzeichnis (toc = table of content) ein
%\addcontentsline{toc}{chapter}{Anhang}
%Einbinden des Anhangs mit sämtlichen Verzeichnissen
%\singlespacing

%!TEX root = ../Bachelorarbeit.tex

\part*{Anhang}
\addcontentsline{toc}{part}{Anhang}
\lhead[ \leftmark   ]{\textbf{Anhang}}
%\addcontentsline{toc}{chapter}{Anhang}

\chapter*{Inhalt Anhang}

\textbf{Auf der mitgelieferten CD befindliche Dateien:}\\
\begin{enumerate}
	\item Video der Ergebnisse des entwickelten Transmovers 
	\item Marktanalyse
		\begin{itemize}
			\item BIC Analyse
			\item Frageb�gen befragter Betroffener
		\end{itemize}
	\item Skizzen und Entw�rfe
		\begin{itemize}
			\item Skizze
			\item Entwurf LEGO Designer
		\end{itemize}
	\item Programmablaufpl�ne
	\item Excel Datei Auswertung Winkelmessung
	\item Sequenzdiagramme
	\item Quellcodes
		\begin{itemize}
			\item Winkeleinmessung
			\item Transmover LabVIEW Programme
			\item Transmover JAVA Programme
			\item Wii Remote Programme
		\end{itemize}
\end{enumerate}

%Zeilenabstand 1 fach für den Eid

%\singlespacing
%Einbinden des Eides
%!TEX root = ../Bachelorarbeit.tex

\chapter*{Erkl�rung}
\addcontentsline{toc}{part}{Erkl�rung}
\lhead[ \leftmark   ]{\textbf{Erkl�rung}}
Ich versichere, die von mir vorgelegte Arbeit selbst�ndig verfasst zu haben.\\ \\
Alle Stellen, die w�rtlich oder sinngem�\ss \ aus ver�ffentlichten Arbeiten anderer entnommen sind, habe ich als entnommen kenntlich gemacht. S�mtliche Quellen und Hilfsmittel, die ich f�r die Arbeit benutzt habe, sind angegeben.\\ \\
Die Arbeit hat nach meinem Wissen mit gleichem Inhalt noch keiner anderen Pr�fungsbeh�rde vorgelegen.
\vspace{1.5cm}
\\
Gummersbach, 14. Oktober 2010
\vspace{2cm}
\\
Thomas Karanatsios


%Fügt die Zeile "`Glossar"' als Chapter ins Inhaltsverzeichnis ein
%\addcontentsline{toc}{chapter}{Glossar}
%Zeilenabstand 1 fach für die Verzeichnisse
%\singlespacing
%Einbindne der Verzeichnisse
%\singlespacing
%Das Glossar ist Optional und mit Tabellen etwas getrickst 
%es geht bestimmt auch anders aber ich hatte leider keine Zeit mehr bei meiner Arbeit das zu machen.
\chapter*{Glossar}
\addcontentsline{toc}{part}{Glossar}
\lhead[ \leftmark   ]{\textbf{Glossar}}
\begin{table}[!ht]	
	 \begin{tabular}{ l p{10cm}}
			\textbf{Autonom} 			& das Programm welches implementiert ist arbeitet weitgehend unabh�ngig von Benutzer eingriffen. \tabularnewline [7pt]
			\textbf{Biometrie} 		& Die Biometrie auch Biometrik genannt besch�ftigt sich mit Messungen an Lebewesen und den dazu erforderlichen Mess- und Auswerteverfahren.\tabularnewline [7pt]
			\textbf{Bionisch} 		& Adjektiv zur Beschreibung eines Organismus, dessen biologische Grundlage durch technische
															M�glichkeiten verbessert wurde.\tabularnewline [7pt]
			\textbf{Degrees of freedom} & (DOF) bedeutung Freiheitsgrade = Der Freiheitsgrad bezeichnet einen Parameter eines Systems. Die Eigenschaft, ein Freiheitsgrad zu sein, ergibt sich f�r einen Parameter
																		daraus, Mitglied in einer Menge von Parametern zu sein, die das System beschreiben.\tabularnewline [7pt]
			\textbf{Deliberativ} 	& Deliberativ = erw�gen, �berlegen, sich
															entscheiden, beschlie\ss en ist eine semantische Funktion Verbmodus des Konjunktivs z. B. im
															Lateinischen (coniunctivus deliberativus), die eine �berlegende R�ckfrage als Reaktion auf eine
															Aufforderung ausdr�ckt.\tabularnewline [7pt]
			\textbf{Dynamik} 				& Eine Dynamik steht f�r, das Teilgebiet der Mechanik, das sich mit der Wirkung von Kr�ften befasst\tabularnewline [7pt]												
			\textbf{Endeffektor} 	& Als Endeffektor wird in der Robotik das letzte Element einer kinematischen Kette bezeichnet. Bei 
															Industrierobotern kann es sich hierbei zum Beispiel um eine Einheit zum Schwei\ss en von
															Autokarosserien oder allgemein um einen einfachen Greifer handeln. Der im englischen als TCP
															(Tool Center Point) bezeichnete ausgezeichnete Punkt am Ende der kinematischen Kette ist das
															Zielsystem, f�r das die aus der gestellten Aufgabe resultierenden Positionierunganforderungen
															gelten. Aufgaben spezifisch kann der TCP dabei auch au\ss erhalb des Roboters liegen, Beispiele
															w�ren der Fokus eines gegriffenen Lasers oder auch die Mitte des gerade transportierten
															Objekts.\tabularnewline [7pt]									
%			\textbf{Exoskelett} 	& Ein Exoskelett ist eine St�tzstruktur f�r einen Organismus, das eine stabile �u\ss ere H�lle um diesen bildet.\tabularnewline [7pt]							
		\end{tabular}
	\vspace{1.0em}
\end{table}


\newpage
\begin{table}[!ht]	
	 \begin{tabular}{ l p{10cm}}
	 		\textbf{Dorsalextension} 	& steht f�r die Bewegung in den Zehengelenken in Richtung Fu\ss r�cken.\tabularnewline [7pt]				
	 		\textbf{Extension} 		& Die Extension (von lat. extensio "Streckung") ist die Streckung eines Gelenkes. Die gegenl�ufige
															Bewegung wird als Flexion bezeichnet.\tabularnewline [7pt]																
			\textbf{Exoskelett} 	& Ein Exoskelett ist eine St�tzstruktur f�r einen Organismus, das eine stabile �u\ss ere H�lle um diesen bildet.\tabularnewline [7pt]					
	 	 	\textbf{Energy Harversting} & Als Energy Harvesting (w�rtlich �bersetzt Energie-Ernten) bezeichnet man die Erzeugung von
																		Strom aus Quellen wie Umgebungstemperatur, Vibrationen oder Luftstr�mungen. Die Industrie
																		entwickelt bereits heute Energiequellen f�r drahtlose Sensornetzwerke oder Anwendungen wie
																		etwa Fernbedienungen an schwer erreichbaren Stellen. Energy Harvesting vermeidet bei
																		Drahtlostechnologien Einschr�nkungen durch kabelgebundene Stromversorgung oder
																		Batterien.\tabularnewline [7pt]			
	 		\textbf{Flexion} 						& Die gegenl�ufige Bewegung zur Extension wird als Flexion bezeichnet.\tabularnewline [7pt]
			\textbf{Inertialsystem} 		& In der Physik ist ein Inertialsystem (von lat. iners "unt�tig, tr�ge") ein Koordinatensystem, in dem sich kr�ftefreie K�rper geradlinig, gleichf�rmig bewegen. In
																		einem Inertialsystem gilt also das newtonsche Tr�gheitsgesetz in seiner einfachsten Form, nach der kr�ftefreie K�rper ihre Geschwindigkeit in Betrag und Richtung
																		beibehalten und Beschleunigungen proportional zur anliegenden Kraft erfolgen. Der Begriff Inertialsystem wurde erstmals von Ludwig Lange (1885) verwendet.\tabularnewline
																		[7pt]
			\textbf{Inhibition} 				& Das Wort Inhibition (lat. inhibere "unterbinden", "anhalten"; veraltend Inhibierung, deutsch Hemmung, Antonym Desinhibition, Desinhibierung) bezeichnet:
																		in der Neurobiologie eine Abnahme der Erregbarkeit von Nervenzellen, siehe Inhibition (Neuron)
																		in der Ethologie die Blockierung einer Verhaltensweise durch innere oder �u\ss ere Faktoren, siehe Bedingte Hemmung
																		in der Digitaltechnik bezeichnet die Inhibition eine Schaltung aus einem UND- und einem NICHT-Glied, siehe Inhibition (Digitaltechnik)\tabularnewline [7pt]
			\textbf{Ipsilateral} 				& Ipsilateral bedeutet "auf derselben K�rperseite oder -h�lfte gelegen".
																		Das Gegenteil von ipsilateral ist kontralateral. \tabularnewline [7pt]
		\end{tabular}
	\vspace{1.0em}
\end{table}

\newpage
%\addcontentsline{toc}{part}{Index}
%\chapter*{Index}
\addcontentsline{toc}{part}{Index}
\lhead[ \leftmark   ]{\textbf{Index}}
\printindex

\newpage
\lhead[ \leftmark   ]{\textbf{Widmung}}
\textbf{\LARGE Widmung}\\\\
Die Widmung ist optional!
Text\\\\

Avec tout l'amoure de mes parents,\\ 
Votre fils

\end{document}
