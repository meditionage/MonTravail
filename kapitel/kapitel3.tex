%!TEX root = ../Montravail.tex

\chapter{Entwicklung des Modells}
\lhead[ \leftmark   ]{\textbf{Modell Realisierung}}
Aufbau siehe Kapitel 1 

\section{Erstes Unterkapitel}
%Beispiel f�r nummerierte Aufz�hlung
\begin{enumerate}
	\item Text\\
	Text
	\item Text\\
	Text
	\item Text\\
	Text
\end{enumerate} 

\newpage
%Eingaben von Source Code
Zuerst erfolgt zum besseren Verst�ndnis die Deklarierung der verwendeten Variablen\\
\begin{lstlisting}[basicstyle=\small,numbers=left,numberstyle=\footnotesize\small,backgroundcolor=\color{sourcegray}]
// die Motoren des rechten Beines werden wie in dem bisher 
// verwendeten Schema zugewiesen 
// H�fte (C), Knie (B) und Fussgelenk (A)
public static Motor motorHip = new Motor(MotorPort.C);
public static Motor motorKnee = new Motor(MotorPort.B);
public static Motor motorAnkle = new Motor(MotorPort.A);
public static boolean sensorReached1 = false;

// der Ultraschallsensor wird an Port S1 erwartet
public static UltrasonicSensor sonicSensor1 = 
              new UltrasonicSensor(SensorPort.S1);

//Der TouchSensor wird an Port S2 erwartet und wird zum iterieren
// durch einen Bewegungsablauf verwendet
public static TouchSensor touchSensor = 
              new TouchSensor(SensorPort.S2);
	
\end{lstlisting}

%\noindent{}Beispiel f�r eine nummerierte Aufz�hlung:
\begin{enumerate}
	\item Item 1
	\item Item 2	
	\item Item 3
\end{enumerate}

%\noindent{}Beispiel f�r eine unnummerierte Aufz�hlung mit neuem Symbol:
\begin{itemize}
\renewcommand{\labelitemi}{$\rightarrow$}
	\item Item 1
	\item Item 2
	\item Item 3
\end{itemize}

%Referenz zu Grafik \ref{fig:showcase} in Kapitel \ref{sec:section}.

