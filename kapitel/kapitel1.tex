%!TEX root = ../Bachelorarbeit.tex

\chapter{Grundlagen}
%\addcontentsline{toc}{chapter}{Grundlagen}
\lhead[ \leftmark   ]{\textbf{Grundlagen}}

%erstes Unterkapitel
\section{Erste �berschrift}

Hier schreiben
%n�chstes unterkapitel ebene 2
\subsection{Zweite �berschrift}
Hier schreiben

%unterkapitel Ebene 3 nicht im Inhaltsverzeichnis aufgef�hrt! 
\subsubsection{Dritte �berschrift}
\textit{hier kann Kursiv geschrieben werden}

%zweites Unterkapitel
\section{Erste �berschrift}

%n�chstes unterkapitel ebene 2
\subsection{Zweite �berschrift}

%Abbildungsbeispiel
%Abbildungsquelle immer! angeben es sei den selber gemacht!!!!
Text....(siehe Abb. 1.1)

\begin{figure}[!ht]
	%mitte der Seite
	\centering
		%[nat�rliche Breite in Pixeln, nat�rliche H�he in Pixeln, Abh�ngigkeit von der Textbreite]
		\includegraphics[natwidth=1200pt, natheight=349pt, width=0.4\textwidth]{grafiken/Robotpeintre.PNG}
	\caption[Aufbau allgemein]{Aufbau und Komponenten von Robotern}
	\label{fig:Aufbau und Komponenten von Robotern}
\end{figure}

%unterkapitel Ebene 3 nicht im Inhaltsverzeichnis aufgef�hrt!
\subsubsection{Dritte �berschrift}

%Aufz�hlung
\begin{itemize}
	\item Die Bewegungsform der Achsen
	\item Anzahl und Anordnung der Achsen
	\item Die Formen des Arbeitsraums
\end{itemize}
    
... Arm zu strecken. (Vgl.\cite{112}) 
%n�chstes unterkapitel ebene 2
\subsection{Zweite �berschrift}
%unterkapitel Ebene 3 nicht im Inhaltsverzeichnis aufgef�hrt!
\subsubsection{Dritte �berschrift}
Hier Text einf�gen\index{einf�gen}

%� ohne weiterf�hrung des Wortes
Roboterfu\ss \ befindet

homogene\\ 4 x 4  Matrix:

\begin{equation} 
T = \begin{pmatrix}
     Ax&Ay&Az&0\\Bx&By&Bz&0\\Cx&Cy&Cz&0\\Px&Py&Pz&1
     \end{pmatrix}
\end{equation}


\begin{equation}(\theta, d, a, \alpha)\end{equation}
 
verschiedene Matrizen:

\begin{equation}
T=\begin{pmatrix}\cos\theta & -\sin\theta \cos\alpha & \sin\theta \sin\alpha & \arccos\theta\\ \sin\theta & \cos\theta \cos\alpha & -\cos\theta \sin\alpha & \arcsin\theta\\ 0 & \sin\alpha & \cos\alpha & d\\ 0 & 0 & 0 & 1\end{pmatrix}
\end{equation}

\begin{equation}
^{n - 1}T_n
  = \begin{pmatrix}
    \cos\theta_n & -\sin\theta_n \cos\alpha_n & \sin\theta_n \sin\alpha_n & a_n \cos\theta_n \\
    \sin\theta_n & \cos\theta_n \cos\alpha_n & -\cos\theta_n \sin\alpha_n & a_n \sin\theta_n \\
    0 & \sin\alpha_n & \cos\alpha_n & d_n \\
    0 & 0 & 0 & 1
  \end{pmatrix}.
\end{equation}


\begin{equation}
T=T_1T_2T_3T_4T_5T_{tcp}
\end{equation}


%Beispiel f�r eine Tabelle
Text...(Siehe Tab. 1.1)
\\\\
\textbf{Titel:}\\%Der Titel bei Tabellen muss immer �ber der Tabelle stehen Statuten FH-Koeln! bei Abbildungen drunter! 
\begin{table}[ht]
\centering
\caption[Titel]{Titel}	
	 \begin{tabular}{|c|p{11cm}|}
			\hline
			\rowcolor{sourcegray}
			\textbf{�berschrift 1} & \textbf{�berschrift 2}\\
			\hline
			Text & Text \newline 
						 Text\\
			\hline
			Text & Text \newline 
						 Text\\
			\hline
		\end{tabular}
	\vspace{1.0em}
	%\caption[Mobilit�tsgrade]{Gliederung der Mobilit�tsgrade}
	\label{tab:Titel}
\end{table}

\newpage
%Beispiel f�r zwei Bilder nebeneinander
\begin{figure}[!ht]
\centering 
\begin{minipage}[hbt]{7cm}
	\includegraphics[width=7cm]{grafiken/HONDA_ASIMO.jpg}
\end{minipage}
\begin{minipage}[hbt]{6.7cm}
	\includegraphics[width=6.7cm]{grafiken/Toyota_Robot_at_Toyota_Kaikan.jpg}
	\end{minipage}
	\caption[Humanoide Roboter]{Humanoide Roboter Links der Aibo von Honda rechts der Roboter von Toyota}
	\label{Humanoide Roboter}
\end{figure}

Bild-Quellen:(\url{http://de.wikipedia.org/wiki/Humanoider_Roboter})\\Sichtung: 17.09.2010\\\\

In Abb. 1.2 sind...Text\\\\
 
%�bersichtshalber besser Begriffe betonen
Kraft \textbf{F}, der Masse \textbf{m} und der Beschleunigung \textbf{a} kann mit der daraus resultieren Formel die Kraft, die wirkt, berechnet werden: \begin{equation}F = m * a \end{equation} Folglich ist die Kraft das Produkt von Masse und Beschleunigung.  

%Formeln 
SI-Einheit der Kraft:
\begin{equation}
[F] = kg * \frac{m}{s^{2}} = Newton (N)
\end{equation}

\begin{equation}M = F * l = F * r * \sin\alpha \end{equation}
 
SI-Einheit des Drehmomentes: \begin{equation}[M] = Newtonmeter (N * m)\end{equation}
 