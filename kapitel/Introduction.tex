\chapter*{Introduction}
\addcontentsline{toc}{chapter}{Introduction}
\lhead[ \leftmark   ]{\textbf{Introduction}}
%!Das Vorwort ist optional.\\ \\

%Manchmal Grau umrandet dies macht man so ich hatte kein Vorwort folglich ist es bei mir auskommentiert
%\noindent{}\fcolorbox{black}{sourcegray}{\parbox{\textwidth}{
%Die Schicksale auf unserer Erde sind teilweise so verschieden und zahlreich, das es kaum möglich sein wird den meisten Menschen Ihr Leben auf irgendeine Weise zu erleichtern oder sogar zu verbessern. 

%dies ist eine Mathematische Formel 
\begin{quotation}
\textit{\enquote{$A\rho\chi\eta\ \ \eta\mu\iota\sigma\upsilon\ \ \pi\alpha\nu\tau o \varsigma$ \\Der Anfang ist die Hälfte des Ganzen.}} (Vgl.\cite{111})
\end{quotation} 

%so wird eine korrekte Fußnote erstellt manche Professoren möchten hier auch Lieteraturverweise sehen
Roboter\footnote{Der Begriff Roboter (tschechisch: robot) wurde von Josef und Karel Capek Anfang des 20. Jahrhunderts durch die Science-Fiction-Literatur geprägt. Der Ursprung liegt im slawischen Wort robota, welches mit Arbeit, Fronarbeit oder Zwangsarbeit übersetzt werden kann. 1921 beschrieb Karel Capek in seinem Theaterstück R.U.R. in Tanks gezüchtete menschenähnliche künstliche Arbeiter. Mit seinem Werk greift Capek das klassische Motiv des Golems auf. Heute würde man Capeks Kunstgeschöpfe als Androiden bezeichnen. Vor der Prägung dieses Begriffes wurden Roboter zum Beispiel in den Werken von Stanislaw Lem als Automaten oder Halbautomaten bezeichnet.}

%ß wird in dieser Arbeit wie folgt gemacht
Bei Rädern und Ketten ist dies hingegen nicht der Fall, Sie brauchen gro\ss flächige Stützpunkte. 

%Zitate werden genormt so gemacht
Das bedeutet, dass letztere Arten der Fortbewegung einen ununterbrochenen Kontakt zum Boden benötigen. (Vgl.\cite{116})\\
%Allerdings fehlt hier noch die Seitenangabe "{Seite 4, ff}" das verlangen nicht alle Professoren so wie meiner 

%für den Index verwendete Wörter werden so angegeben
Dadurch wurden die im Verlauf der Evolution optimierten Konstruktionen von Beinen, das Zusammenspiel von Sensorik\index{Sensorik} und Aktorik\index{Aktorik} und die Steuerung von Gehbewegungen weitestgehend analysiert und dienten somit dem besseren Verständnis der Laufmotorik\footnote{Die Laufmotorik beinhaltet die Bewegungsfunktion und deren Lehre, die Fähigkeit des Körpers sich kontrolliert zu bewegen, die Gesamtheit der vom zentralen Nervensystem kontrollierten Bewegungen des Körpers im Gegensatz zu den unwillkürlichen Reflexen des Körpers und die Unterscheidung in Grob- und Feinmotorik}\index{Laufmotorik}.\\\\ 

An dieser Stelle sei darauf hingewiesen, dass dieses Forschungsgebiet einen stark interdisziplinären Charakter besitzt...


%Falls eine neue Seite erzwungen werden soll macht man dies so
\newpage
%um den Text Fett zu schreiben wir dieser Befehl verwendet
\textbf{Kapitelübersicht}\\\\ 