%!TEX root = ../Montravail.tex

\chapter{algebraic design}
\addcontentsline{toc}{chapter}{xyhas to be defined}
\lhead[ \leftmark   ]{\textbf{xyhas to be defined}}
%!Das Vorwort ist optional.\\ \\

%Manchmal Grau umrandet dies macht man so ich hatte kein Vorwort folglich ist es bei mir auskommentiert
%\noindent{}\fcolorbox{black}{sourcegray}{\parbox{\textwidth}{
%Die Schicksale auf unserer Erde sind teilweise so verschieden und zahlreich, das es kaum m�glich sein wird den meisten Menschen Ihr Leben auf irgendeine Weise zu erleichtern oder sogar zu verbessern. 

%dies ist eine Mathematische Formel 
\begin{quotation}
\textit{\enquote{$A\rho\chi\eta\ \ \eta\mu\iota\sigma\upsilon\ \ \pi\alpha\nu\tau o \varsigma$ \\Der Anfang ist die H�lfte des Ganzen.}} (Vgl.\cite{111})
\end{quotation} 

%so wird eine korrekte Fu�note erstellt manche Professoren m�chten hier auch Lieteraturverweise sehen
Roboter\footnote{Der Begriff Roboter (tschechisch: robot) wurde von Josef und Karel Capek Anfang des 20. Jahrhunderts durch die Science-Fiction-Literatur gepr�gt. Der Ursprung liegt im slawischen Wort robota, welches mit Arbeit, Fronarbeit oder Zwangsarbeit �bersetzt werden kann. 1921 beschrieb Karel Capek in seinem Theaterst�ck R.U.R. in Tanks gez�chtete menschen�hnliche k�nstliche Arbeiter. Mit seinem Werk greift Capek das klassische Motiv des Golems auf. Heute w�rde man Capeks Kunstgesch�pfe als Androiden bezeichnen. Vor der Pr�gung dieses Begriffes wurden Roboter zum Beispiel in den Werken von Stanislaw Lem als Automaten oder Halbautomaten bezeichnet.}

\section{Parser combinators vs. parser generators}
%� wird in dieser Arbeit wie folgt gemacht
introduce a design approach called algebraic design \\\\
Parser combinators vs. parser generators

%Zitate werden genormt so gemacht
Das bedeutet, dass letztere Arten der Fortbewegung einen ununterbrochenen Kontakt zum Boden ben�tigen. (Vgl.\cite{116})\\
%Allerdings fehlt hier noch die Seitenangabe "{Seite 4, ff}" das verlangen nicht alle Professoren so wie meiner 

%f�r den Index verwendete W�rter werden so angegeben
Dadurch wurden die im Verlauf der Evolution optimierten Konstruktionen von Beinen, das Zusammenspiel von Sensorik\index{Sensorik} und Aktorik\index{Aktorik} und die Steuerung von Gehbewegungen weitestgehend analysiert und dienten somit dem besseren Verst�ndnis der Laufmotorik\footnote{Die Laufmotorik beinhaltet die Bewegungsfunktion und deren Lehre, die F�higkeit des K�rpers sich kontrolliert zu bewegen, die Gesamtheit der vom zentralen Nervensystem kontrollierten Bewegungen des K�rpers im Gegensatz zu den unwillk�rlichen Reflexen des K�rpers und die Unterscheidung in Grob- und Feinmotorik}\index{Laufmotorik}.\\\\ 

An dieser Stelle sei darauf hingewiesen, dass dieses Forschungsgebiet einen stark interdisziplin�ren Charakter besitzt...


%Falls eine neue Seite erzwungen werden soll macht man dies so
\newpage
%um den Text Fett zu schreiben wir dieser Befehl verwendet
\textbf{Kapitel�bersicht}\\\\ 
%Falls eine neue Seite erzwungen werden soll macht man dies so
\newpage
\section{SBT build system}
build and run your Scala code
using sbt, the build tool for Scala, and/or an IDE like IntelliJ or Eclipse. See the book's source code repo on GitHub for more information on getting set up with sbt.
Sbt is very smart about ensuring only the minimum number of files are recompiled when changes are made. It also has a number of other nice features which we won't
discuss here.


sbt uses a small number of concepts to support **flexible and powerful build definitions**. There are not that many concepts, but sbt is not exactly like other build systems and there are details you�will�stumble on if you haven?t read the documentation