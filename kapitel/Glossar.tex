%Das Glossar ist Optional und mit Tabellen etwas getrickst 
%es geht bestimmt auch anders aber ich hatte leider keine Zeit mehr bei meiner Arbeit das zu machen.
\chapter*{Glossar}
\addcontentsline{toc}{part}{Glossar}
\lhead[ \leftmark   ]{\textbf{Glossar}}
\begin{table}[!ht]	
	 \begin{tabular}{ l p{10cm}}
			\textbf{Autonom} 			& das Programm welches implementiert ist arbeitet weitgehend unabh�ngig von Benutzer eingriffen. \tabularnewline [7pt]
			\textbf{Biometrie} 		& Die Biometrie auch Biometrik genannt besch�ftigt sich mit Messungen an Lebewesen und den dazu erforderlichen Mess- und Auswerteverfahren.\tabularnewline [7pt]
			\textbf{Bionisch} 		& Adjektiv zur Beschreibung eines Organismus, dessen biologische Grundlage durch technische
															M�glichkeiten verbessert wurde.\tabularnewline [7pt]
			\textbf{Degrees of freedom} & (DOF) bedeutung Freiheitsgrade = Der Freiheitsgrad bezeichnet einen Parameter eines Systems. Die Eigenschaft, ein Freiheitsgrad zu sein, ergibt sich f�r einen Parameter
																		daraus, Mitglied in einer Menge von Parametern zu sein, die das System beschreiben.\tabularnewline [7pt]
			\textbf{Deliberativ} 	& Deliberativ = erw�gen, �berlegen, sich
															entscheiden, beschlie\ss en ist eine semantische Funktion Verbmodus des Konjunktivs z. B. im
															Lateinischen (coniunctivus deliberativus), die eine �berlegende R�ckfrage als Reaktion auf eine
															Aufforderung ausdr�ckt.\tabularnewline [7pt]
			\textbf{Dynamik} 				& Eine Dynamik steht f�r, das Teilgebiet der Mechanik, das sich mit der Wirkung von Kr�ften befasst\tabularnewline [7pt]												
			\textbf{Endeffektor} 	& Als Endeffektor wird in der Robotik das letzte Element einer kinematischen Kette bezeichnet. Bei 
															Industrierobotern kann es sich hierbei zum Beispiel um eine Einheit zum Schwei\ss en von
															Autokarosserien oder allgemein um einen einfachen Greifer handeln. Der im englischen als TCP
															(Tool Center Point) bezeichnete ausgezeichnete Punkt am Ende der kinematischen Kette ist das
															Zielsystem, f�r das die aus der gestellten Aufgabe resultierenden Positionierunganforderungen
															gelten. Aufgaben spezifisch kann der TCP dabei auch au\ss erhalb des Roboters liegen, Beispiele
															w�ren der Fokus eines gegriffenen Lasers oder auch die Mitte des gerade transportierten
															Objekts.\tabularnewline [7pt]									
%			\textbf{Exoskelett} 	& Ein Exoskelett ist eine St�tzstruktur f�r einen Organismus, das eine stabile �u\ss ere H�lle um diesen bildet.\tabularnewline [7pt]							
		\end{tabular}
	\vspace{1.0em}
\end{table}


\newpage
\begin{table}[!ht]	
	 \begin{tabular}{ l p{10cm}}
	 		\textbf{Dorsalextension} 	& steht f�r die Bewegung in den Zehengelenken in Richtung Fu\ss r�cken.\tabularnewline [7pt]				
	 		\textbf{Extension} 		& Die Extension (von lat. extensio "Streckung") ist die Streckung eines Gelenkes. Die gegenl�ufige
															Bewegung wird als Flexion bezeichnet.\tabularnewline [7pt]																
			\textbf{Exoskelett} 	& Ein Exoskelett ist eine St�tzstruktur f�r einen Organismus, das eine stabile �u\ss ere H�lle um diesen bildet.\tabularnewline [7pt]					
	 	 	\textbf{Energy Harversting} & Als Energy Harvesting (w�rtlich �bersetzt Energie-Ernten) bezeichnet man die Erzeugung von
																		Strom aus Quellen wie Umgebungstemperatur, Vibrationen oder Luftstr�mungen. Die Industrie
																		entwickelt bereits heute Energiequellen f�r drahtlose Sensornetzwerke oder Anwendungen wie
																		etwa Fernbedienungen an schwer erreichbaren Stellen. Energy Harvesting vermeidet bei
																		Drahtlostechnologien Einschr�nkungen durch kabelgebundene Stromversorgung oder
																		Batterien.\tabularnewline [7pt]			
	 		\textbf{Flexion} 						& Die gegenl�ufige Bewegung zur Extension wird als Flexion bezeichnet.\tabularnewline [7pt]
			\textbf{Inertialsystem} 		& In der Physik ist ein Inertialsystem (von lat. iners "unt�tig, tr�ge") ein Koordinatensystem, in dem sich kr�ftefreie K�rper geradlinig, gleichf�rmig bewegen. In
																		einem Inertialsystem gilt also das newtonsche Tr�gheitsgesetz in seiner einfachsten Form, nach der kr�ftefreie K�rper ihre Geschwindigkeit in Betrag und Richtung
																		beibehalten und Beschleunigungen proportional zur anliegenden Kraft erfolgen. Der Begriff Inertialsystem wurde erstmals von Ludwig Lange (1885) verwendet.\tabularnewline
																		[7pt]
			\textbf{Inhibition} 				& Das Wort Inhibition (lat. inhibere "unterbinden", "anhalten"; veraltend Inhibierung, deutsch Hemmung, Antonym Desinhibition, Desinhibierung) bezeichnet:
																		in der Neurobiologie eine Abnahme der Erregbarkeit von Nervenzellen, siehe Inhibition (Neuron)
																		in der Ethologie die Blockierung einer Verhaltensweise durch innere oder �u\ss ere Faktoren, siehe Bedingte Hemmung
																		in der Digitaltechnik bezeichnet die Inhibition eine Schaltung aus einem UND- und einem NICHT-Glied, siehe Inhibition (Digitaltechnik)\tabularnewline [7pt]
			\textbf{Ipsilateral} 				& Ipsilateral bedeutet "auf derselben K�rperseite oder -h�lfte gelegen".
																		Das Gegenteil von ipsilateral ist kontralateral. \tabularnewline [7pt]
		\end{tabular}
	\vspace{1.0em}
\end{table}